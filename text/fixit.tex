\documentclass{beamer}
\usepackage[ngerman]{babel}
\usepackage[latin1]{inputenc}
\usepackage{helvet}

\usetheme{Frankfurt} % Warsaw is nice, too


\title[]{Fix-It}
\subtitle{How to repair your electronic stuff when it's broken}
\author{Andreas M�ller}
\date{}
\institute{Cosin 2009}

\AtBeginSection[]{\frame{\tableofcontents[current,sectionstyle=show/shaded,subsectionstyle=show/show/hide]}}
\setbeamercovered{transparent=50}

\begin{document}

\frame[plain]{\titlepage}

\setcounter{tocdepth}{2}
\frame[plain]{\tableofcontents}

\section{Introduction}

\subsection{Disclaimer}
\begin{frame}
	\frametitle{Disclaimer}
	\begin{block}{Please note:}
		\begin{itemize}
			\item Erfahrungswerte
			\item Use at your own risk \ldots
			\item Ger�t kann besch�digt werden
			\item Mensch kann besch�digt werden
				\begin{itemize}
					\item Ger�t immer ausstecken
					\item Kondensatoren entladen!
					\item Verstand benutzen
				\end{itemize}
			\item Brandgefahr
				\begin{itemize}
					\item Keine Sicherungen �berbr�cken oder st�rkere
						einbauen
					\item Keine brennbaren Materialien verwenden (Isolierband,
						etc)
					\item Brennbarkeit notfalls kurz mit Feuerzeug testen
				\end{itemize}
			\item Voids your warranty
		\end{itemize}
	\end{block}
\end{frame}


\subsection{Why repair?}
\begin{frame}
	\frametitle{Why repair?}
	\begin{block}{Why repair stuff?}
		\begin{itemize}
			\item Neugier
			\item Herausforderung/pers�nlicher Ehrgeiz
			\item Billiger
			\item Umweltfreundlicher
			\item Geht oft schneller als altes Ger�t entsorgen und neues
				kaufen 
			\item Weniger nervt�tend, als sich mit \$KUNDENSERVICE
				rumzuschlagen
		\end{itemize}
	\end{block}
\end{frame}

\subsection{Ausr�stung}
\begin{frame}
	\frametitle{Ausr�stung}
	\begin{block}{Equipment}
		\begin{itemize}
			\item Allgemeine Werkstattausr�stung (Schraubenzieher, etc.)
			\item Multimeter
				\begin{itemize}
					\item Akustischer Durchgangspr�fer ist praktisch
					\item Kosten: 20-30 CHF
				\end{itemize}
			\item L�tkolben
				\begin{itemize}
					\item Temperaturgeregelte Elektronikl�tstation von
						Vorteil (Temperatur muss nicht einstellbar sein)
					\item Kosten: ca 150 CHF 
				\end{itemize}
			\item Kleinteile (L�tzinn, Entl�tlitze, Kabel, Ersatzteile, etc)
				\begin{itemize}
					\item ca 50 CHF 
				\end{itemize}
		\end{itemize}
	\end{block}
\end{frame}

\begin{frame}
	\frametitle{Ausr�stung / Bauteile}
	\begin{block}{Bezugsquellen f�r Ausr�stung und Bauteile}
		\begin{itemize}
			\item Online: distrelec.ch 
			\item Online: ebay.ch
			\item Offline/ZH: Bastli (see www.bastli.ethz.ch)
			\item Offline/ZH: Pusterla (see pusterla.ch)
			\item Samples
			\item etc. 
		\end{itemize}
	\end{block}
\end{frame}

\section{Was geht kaputt / wie reparieren?}

\subsection{Sicherungen}
\begin{frame}
	\frametitle{Sicherungen}
	\begin{block}{Was?}
		\begin{itemize}
			\item Selbstschutz
			\item Geht oft als erstes kaputt (soll auch!)
			\item Vorhanden v.a. in gr�sseren/teureren Ger�ten
		\end{itemize}
	\end{block}
	\begin{block}{Wie erkennen?}
		\begin{itemize}
			\item Ger�t verh�lt sich, als h�tte es (teilweise) keinen Strom
			\item Am besten alle Sicherungen mit Durchgangspr�fer testen
			\item Sicherungen oft mit F beschriftet (F1, F2, ..)
		\end{itemize}
	\end{block}
	\begin{block}{Wie reparieren?}
		\begin{itemize}
			\item Sicherung mit gleicher Stromgrenze verwenden (!)
			\item Kosten: ca CHF 0.50
		\end{itemize}
	\end{block}
\end{frame}
\begin{frame}[plain]
	\begin{center}
		\includegraphics[scale=0.4]{photos/sicherungen.jpg}
	\end{center}
\end{frame}
\begin{frame}[plain]
	\begin{center}
		\includegraphics[scale=0.3]{photos/lcd_sicherungen.jpg}
	\end{center}
\end{frame}
\begin{frame}[plain]
	\begin{center}
		\includegraphics[scale=0.3]{photos/lcd_sicherungen_close1.jpg}
	\end{center}
\end{frame}
\begin{frame}[plain]
	\begin{center}
		\includegraphics[scale=0.3]{photos/lcd_sicherungen_close2.jpg}
	\end{center}
\end{frame}

\subsection{Kondensatoren}
\begin{frame}
	\frametitle{Kondensatoren}
	\begin{block}{Wo?}
		\begin{itemize}
			\item Elektrolytkondensatoren in Netzteilen
			\item Hochspannungskondensatoren (e.g. Invertereinheit)
		\end{itemize}
	\end{block}
	\begin{block}{Wie erkennen?}
		\begin{itemize}
			\item Geruch, Brandspuren, Knall beim Kaputtgehen
			\item Ausfall von Elkos oft wegen Alterung
			\item Kondensatoren oft mit C beschriftet (C1, C2, ..)
		\end{itemize}
	\end{block}
	\begin{block}{Wie reparieren?}
		\begin{itemize}
			\item gleicher Typ (z.B. Elko)
			\item gleiche Kapazit�t
			\item gleiche oder h�here Spannung
		\end{itemize}
	\end{block}
\end{frame}
\begin{frame}[plain]
	\begin{center}
		\includegraphics[scale=0.3]{photos/kondensatoren.jpg}
	\end{center}
\end{frame}
\begin{frame}[plain]
	\begin{center}
		\includegraphics[scale=0.3]{photos/kondensator_futsch.jpg}
	\end{center}
\end{frame}
\begin{frame}[plain]
	\begin{center}
		\includegraphics[scale=0.3]{photos/inverter.jpg}
	\end{center}
\end{frame}
\begin{frame}[plain]
	\begin{center}
		\includegraphics[scale=0.3]{photos/inverter_close.jpg}
	\end{center}
\end{frame}

\subsection{Mechanisches}
\begin{frame}
	\frametitle{Mechanische Elemente und Kontakte}
	\begin{block}{Wo?}
		\begin{itemize}
			\item Ger�te, die viel bewegt werden
			\item L�fter, Kabel, Batteriekontakte, Kopfh�rerkontakt
		\end{itemize}
	\end{block}
	\begin{block}{Wie erkennen?}
		\begin{itemize}
			\item Defekt oft sichtbar
			\item Mit Voltmeter und/oder Durchgangspr�fer
		\end{itemize}
	\end{block}
	\begin{block}{Wie reparieren?}
		\begin{itemize}
			\item L�tkolben + Draht :) 
		\end{itemize}
	\end{block}
\end{frame}

\subsection{Interne Batterien und Akkus}
\begin{frame}
	\frametitle{Interne Batterien und Akkus}
	\begin{block}{Was?}
		\begin{itemize}
			\item St�tzbatterie f�r Uhr oder fl�chtigen Speicher
		\end{itemize}
	\end{block}
	\begin{block}{Wo?}
		\begin{itemize}
			\item Motherboards, MP3 Player, Kameras, etc.
		\end{itemize}
	\end{block}
	\begin{block}{Wie erkennen?}
		\begin{itemize}
			\item Uhr verliert die Zeit nach Ausschalten
			\item Settings nach Ausschalten auf Default (z.B. BIOS)
			\item Oft Alterserscheinung
			\item Manchmal sichtbar (Batterie ausgelaufen, etc.)
		\end{itemize}
	\end{block}
	\begin{block}{Wie reparieren?}
		\begin{itemize}
			\item Gleiche Spannung und Typ; notfalls Ausbauen ohne Ersatz
		\end{itemize}
	\end{block}
\end{frame}

\subsection{Transistoren}
\begin{frame}
	\frametitle{Transistoren}
	\begin{block}{Wo?}
		\begin{itemize}
			\item z.B. Leistungstransistoren in regelbaren Netzteilen
		\end{itemize}
	\end{block}
	\begin{block}{Wie erkennen?}
		\begin{itemize}
			\item Ger�t geht w�hrend Betrieb kaputt, ohne Schall/Rauch
			\item Ger�t funktioniert normal, aber keine Ausgangsspannung
			\item 3 oder mehr Anschl�sse
			\item Testen mit Diodentester
		\end{itemize}
	\end{block}
	\begin{block}{Wie reparieren?}
		\begin{itemize}
			\item Gleiches Modell oder Ersatzmodell (Google hilft ggf.)
		\end{itemize}
	\end{block}
\end{frame}
\begin{frame}[plain]
	\begin{center}
		\includegraphics[scale=0.3]{photos/transistoren.jpg}
	\end{center}
\end{frame}

\section{Emergencies}
\subsection{Wasser}
\begin{frame}
	\frametitle{Notf�lle -- Ger�t wurde nass}
	\begin{block}{Was machen?}
		\begin{itemize}
			\item die Bauteile selbst haben normalerweise kein Problem mit
				Wasser, aber wenn Strom fliesst, entstehen Kurzschl�sse 
			\item auch reines Wasser leitet Strom!
			\item nach Kontakt mit Wasser sofort alle Stromquellen entfernen
			\item ggf. vorsichtig putzen, z.B. mit Wasser oder Reinbenzin
			\item vor erneuter Inbetriebnahme \emph{komplett} trocknen
		\end{itemize}
	\end{block}
\end{frame}

\section{Fazit}
\subsection{Fazit}
\begin{frame}
	\frametitle{Fazit}
	\begin{block}{Fazit}
		\begin{itemize}
			\item Reparatur oft m�glich, ohne die Schaltung zu verstehen
			\item Kosten sind meist minimal
			\item Faustregel: Abnutzung entsteht, wo Energie umgesetzt wird
		\end{itemize}
	\end{block}
\end{frame}

\section{Questions}
\subsection{Questions}
\begin{frame}
	\frametitle{Questions?}
	\begin{block}{Fragen}
		Fragen?
	\end{block}
	\begin{block}{Links:}
		\begin{itemize}
			\item \small{\url{http://de.wikipedia.org/wiki/Kategorie:Elektrisches\_Bauelement}}
			\item \small{\url{http://www.elektronik-kompendium.de}}
			\item \small{\url{http://people.ee.ethz.ch/~andrmuel/files/uncompressed/ClockDoc.pdf}}
		\end{itemize}
	\end{block}
\end{frame}

\end{document}

