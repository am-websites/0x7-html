\documentclass[10pt]{beamer}
\usepackage[english]{babel}
\usepackage[latin1]{inputenc}
\usepackage{helvet}

\usetheme{Frankfurt} % Warsaw is nice, too


\title[]{DAB}
\subtitle{Software Receiver Implementation}
\author{Andreas M�ller \\ Supervisor: Michael Lerjen}
\date{June 13, 2008}
\institute{ETH -- ITET -- CTL}

\AtBeginSection[]{\frame{\tableofcontents[current,sectionstyle=show/shaded,subsectionstyle=show/show/hide]}}
\setbeamercovered{transparent=50}

\begin{document}

\frame[plain]{\titlepage}

\setcounter{tocdepth}{2}
\frame[plain]{\tableofcontents}

\section{Introduction}

\subsection{Task}

\begin{frame}
	\frametitle{Task}

	\begin{large}	
		Goal: Implementation of a real-time DAB receiver as SDR
	\end{large}

	\begin{center}
		\begin{minipage}{8cm}
			\begin{block}{SDR}
				Software Defined Radio $\rightarrow$ (almost) all signal processing in software
			\end{block}
			\begin{block}{DAB}
				Digital Audio Broadcasting $\rightarrow$ digital radio
				technology standardized by ETSI
			\end{block}
			\begin{block}{Real-time}
				Process data as fast as it arrives $\rightarrow$ 2 MSPS or 16 MB/s
			\end{block}
		\end{minipage}
	\end{center}

\end{frame}

\subsection{Software Defined Radio}

\begin{frame}
	\frametitle{Software Defined Radio}
	\begin{block}{Idea}
		Digitize the signal and do all the signal processing in (high level,
		architecture independent) software.
	\end{block}
	\begin{block}{Strengths}
		\begin{itemize}
			\item Flexibility
			\item Reusable code, fast development cycle
			\item Cognitive radio: Adapts itself dynamically to RF environment
				\\$\rightarrow$ better spectral and power efficiency 
		\end{itemize}
	\end{block}
	\begin{block}{Weaknesses}
		\begin{itemize}
			\item Limited sample rate and dynamic range of ADCs and DACs
				\\$\rightarrow$ analog front end needed for filtering
			\item Resource usage, energy consumption, cost
		\end{itemize}
	\end{block}
\end{frame}


\subsection{DAB}

\begin{frame}
	\frametitle{Digital Audio Broadcasting (DAB) Specification}
	\begin{block}{Modes}
		Four modes  for different frequency ranges and RF characteristics
		\begin{itemize}
			\item Presentation: Mode I (Code: All Modes)
		\end{itemize}
	\end{block}
	\begin{block}{DAB Mode I OFDM signal}
		\begin{itemize}
			\item Frames with 76 OFDM symbols (1 pilot, 75 data)
			\item Null symbols (energy zero) to separate frames
			\item 1536 subcarriers � 1 kHz \&  central carrier zero $\to$ 1.537 MHz
			\item D-QPSK modulation for each subcarrier
			\item Cyclic prefix: 504 samples $\to$ SFN with max. TX distance 74 km
		\end{itemize}
	\end{block}
	\begin{block}{Upper Layers}
		\begin{itemize}
			\item Punctured convolutional coding
			\item Energy dispersal, Time interleaving
			\item MPEG 2 audio coding
		\end{itemize}
	\end{block}
\end{frame}

\subsection{GNU Radio and USRP}

\begin{frame}
	\frametitle{GNU Radio}
	\begin{block}{Overview}
		\begin{itemize}
			\item Open source framework for real-time software radios
			\item Provides many common building blocks: FFT, FIR \& IIR
				filters, mathematical operations, AGC, modulation \&
				demodulation, \ldots
		\end{itemize}
	\end{block}
	\begin{block}{Flow Graph Concept}
		\begin{itemize}
			\item Programmer creates a directed graph for sample flow
			\item Signal processing blocks are written in C++ and wired together in Python
		\end{itemize}
	\end{block}
	\begin{block}{Signal Processing Block}
		\begin{itemize}
			\item \texttt{work()} function receives a number of samples from
				scheduler
			\item Block processes as many samples as possible and returns the
				number of consumed and produced samples
		\end{itemize}
	\end{block}
\end{frame}


\begin{frame}
	\frametitle{Universal Software Radio Peripheral (USRP)}
	\begin{block}{Hardware}
		\begin{itemize}
			\item Interface between computer and antenna is needed
			\item Most commonly used with GNU Radio: USRP
		\end{itemize}
	\end{block}
	\begin{block}{USRP}
		\begin{itemize}
			\item Two AD9862 Mixed Signal Front-End Processors
				\begin{itemize}
					\item 4 DACs with sampling rate 128 MSPS $\to$ 2 I/Q TX channels
					\item 4 ADCs with sampling rate 64 MSPS $\to$ 2 I/Q RX channels
				\end{itemize}
			\item Altera Cyclone FPGA for conversion to/from baseband,
				decimation/interpolation, multiplexing and buffering
			\item Cypress FX2 USB 2.0 interface
			\item Daughterboards according to selected frequency range
		\end{itemize}
	\end{block}
\end{frame}

\begin{frame}[plain]
	\begin{center}
		\includegraphics[scale=0.25]{pics/usrp_ettus.jpg}\\
		\small (Source: http://ettus.com)
	\end{center}
\end{frame}

\section{Implementation}
\subsection{OFDM Synchronisation}
\begin{frame}
	\frametitle{OFDM I -- Synchronisation}
	\begin{block}{Time Synchronisation}
		\begin{itemize}
			\item Frame start detection must be accurate, as the other blocks
				depend on it
			\item Can easily be done by looking at the energy of the signal
				(Null symbols)
			\item Implemented with moving sum, inverter and peak detector
		\end{itemize}
	\end{block}
	\begin{block}{Frequency Synchronisation}
		\begin{itemize}
			\item Small subcarrier spacing $\to$ accurate synchronisation needed
			\item Fine frequency synchronisation (offsets $<$ subcarrier spacing)
				\begin{itemize}
					\item compare cyclic prefix to end of the symbol $\to$
						fine frequency offset can be estimated from the phase
						offset
				\end{itemize}
			\item Coarse frequency synchronisation (offsets $>$ subcarrier spacing)
				\begin{itemize}
					\item done after fine frequency synchronisation and after FFT
					\item simply shift signal in the frequency domain $\to$ very efficient
				\end{itemize}
		\end{itemize}
	\end{block}
\end{frame}

\subsection{OFDM Demodulation}
\begin{frame}
	\frametitle{OFDM II -- Demodulation}
	\begin{block}{Demodulation}	
		\begin{itemize}
			\item Besides time and frequency synchronisation,
				demodulation is rather straightforward
			\item Sampler: Remove cyclic prefix, pack each OFDM symbol in a
				vector
			\item FFT
			\item Calculate phase difference (undo the D in D-QPSK) 
			\item Magnitude equalization (only needed for soft bits, as the
				information is only in the phase)
			\item Undo frequency interleaving: Mix symbols according to
				sequence specified in DAB standard
			\item I and Q components contain independent bits $\to$ simply
				check if $\Re(x)>0$ and $\Im(x)>0$
		\end{itemize}
	\end{block}
\end{frame}

\section{Evaluation}

\subsection{Test Setup}

\begin{frame}
	\frametitle{Test Setup}
	\begin{block}{Simulation Cycle}
		\begin{itemize}
			\item Generate random bytes
			\item Modulation
			\item Channel-model distorts OFDM signal
			\item Demodulation
			\item Calculate BER from original and received bytes
		\end{itemize}
	\end{block}
	\begin{block}{Channel Model}
		\begin{itemize}
			\item Sampling frequency offset modeled by fractional interpolator
			\item Multipath propagation modeled with FIR filter
			\item Frequency offset (signal source + multiplication block)
			\item AWGN (noise source + adder block)
		\end{itemize}
	\end{block}
\end{frame}

\subsection{Results}

\begin{frame}
	\frametitle{Results -- SNR}	
	\begin{center}
		\includegraphics[scale=0.5]{../report/plots/snr_ber.png}
	\end{center}
\end{frame}


\begin{frame}
	\frametitle{Results -- Effects of Multipath Propagation}	
	\begin{center}
		\includegraphics[scale=0.4]{../report/plots/multipath_500kb.png}
	\end{center}
	\begin{center}
		\begin{footnotesize}
			\begin{tabular}{|l|c|c|c|c|}
				\hline 
				DAB Mode             & 1   & 2   & 3  & 4 \\
				\hline
				Cyclic Prefix Length & 504 & 126 & 63 & 252 \\
				\hline 
			\end{tabular}
		\end{footnotesize}
	\end{center}
\end{frame}

\section{Conclusions}
\begin{frame}
	\frametitle{Conclusions}

	\begin{block}{Conclusions}
		\begin{itemize}
			\item Real-time processing is possible
			\item FIBs successfully decoded 
			\item No audio yet
		\end{itemize}
	\end{block}
	\begin{block}{Challenges}
		\begin{itemize}
			\item Very efficient algorithms and programming needed
			\item Many signal processing papers are written from a primarily
				mathematical perspective
		\end{itemize}
	\end{block}
	\begin{block}{Advantages}
		\begin{itemize}
			\item Same code for simulation and actual receiver 
			\item Open source code of existing blocks helps understand algorithms
			\item Existing code can sometimes be adapted for new purposes
			\item GNU Radio: Large and enthusiastic community
		\end{itemize}
	\end{block}
\end{frame}

\section{Questions}
\begin{frame}
	\frametitle{Questions?}
	\begin{center}
		\large{Thank you for your attention.}
	\end{center}
\end{frame}

\end{document}

