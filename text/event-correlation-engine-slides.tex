\documentclass[10pt]{beamer}
\usepackage[ngerman]{babel}
\usepackage[utf8]{inputenc}
\usepackage{helvet}
\usepackage{listings}

\usetheme{Frankfurt}

% listing options
\definecolor{lightgray}{gray}{0.9}
\lstset{basicstyle=\footnotesize, numbers=left, numberstyle=\tiny, backgroundcolor=\color{lightgray}, showstringspaces=false}


\title[]{Event Correlation Engine}
\subtitle{Master's Thesis -- Final Presentation}
\author{Andreas Müller\\[0.8em]Tutors:\\Christoph Göldi, Bernhard Tellenbach\\[0.8em]Supervisor:\\Prof. B. Plattner}
% \institute{ETH~~--~~ITET~~--~~CSG}
\institute{}
% \date{September 2, 2009}
\date{}
\titlegraphic{
	\begin{minipage}{0.7\textwidth}
		\flushleft{
		\includegraphics[scale=0.2]{../report/figures/logos/open_systems.pdf}	
		}
		\flushright{
		\vspace{-1.75cm}
		\includegraphics[scale=0.55]{../report/figures/logos/tik.pdf}
		}
	\end{minipage}
	\vspace*{1cm}
}

\AtBeginSection[]{\frame{\tableofcontents[current,sectionstyle=show/shaded,subsectionstyle=show/show/hide]}}
\setbeamercovered{transparent=50}

\begin{document}

\begin{frame}[plain]
	\titlepage
\end{frame}

\setcounter{tocdepth}{1}
\frame[plain]{\tableofcontents}
\setcounter{tocdepth}{2}

\section{Introduction}
\subsection{Event Correlation}
\begin{frame}
	\frametitle{Event Correlation}
	\begin{block}{Event Correlation}
		\begin{itemize}
			\item Event
				\begin{itemize}
					\item Any occurrence; anything, which happened
					\item Computing: message, what happened when
				\end{itemize}
			\item Correlation: analysis of co-relations
			\item Goal: gain higher-level knowledge
			\item Applications
				\begin{itemize}
					\item Market data analysis
					\item Algorithmic trading
					\item Fraud detection
					\item System log analysis
					\item Network management
				\end{itemize}
		\end{itemize}
	\end{block}
	\begin{block}{Event Correlation Engine (ECE)}
		\begin{itemize}
			\item Application or toolkit to correlate events 
		\end{itemize}
	\end{block}
\end{frame}

\section{Motivation and Task}

\subsection{Background}
\begin{frame}
	\frametitle{Background}
	\begin{block}{Background}
		\begin{itemize}
			\item Master's thesis at Open Systems AG
			\item Open Systems provides managed security for customers around
				the world, operating a global network with more than 1500
				hosts 
		\end{itemize}
	\end{block}
	\begin{block}{Setup}
		\begin{itemize}
			\item Hosts generate events from syslog messages and network
				observations, e.g.:
				\begin{itemize}
					\item Network link down
					% \item network link up
					% \item error updating anti-virus signatures
					\item CPU load high
				\end{itemize}
			\item Events end up in tickets, which are handled manually
		\end{itemize}
	\end{block}
\end{frame}

\subsection{Motivation and Task}
\begin{frame}
	\frametitle{Motivation and Task}
	\begin{block}{Problems}
		\begin{itemize}
			\item Handling all events manually is time consuming and
				cumbersome
			\item Simple problems create too many events, important events may
				be overlooked
			\item Same problem has to be handled again and again
		\end{itemize}
	\end{block}
	\begin{block}{How to mitigate these problems?}
		\begin{itemize}
			\item[$\Rightarrow$] Intelligently pre-process the events with an
				ECE
		\end{itemize}
	\end{block}
	\begin{block}{Task}
		\begin{itemize}
			\item Choose or design, implement and evaluate an ECE suitable to
				extend the ticketing system of Open Systems
		\end{itemize}
	\end{block}
\end{frame}

% \subsection{Task}
% \begin{frame}
	% \frametitle{Task}
	% \begin{block}{Goals for the ECE}
		% \begin{itemize}
			% \item Make tickets more intelligible by collecting related events
			% \item Help find a problems root-cause faster by structuring the
				% tickets
			% \item Escalate important events; suppress uncritical events
		% \end{itemize}
	% \end{block}
% \end{frame}

\section{Approach}

\subsection{Event Pattern Analysis}

\begin{frame}
	\frametitle{Event Pattern Analysis}
	\begin{block}{Pattern Analysis: Goals}
		\begin{itemize}
			\item Qualitative and quantitative overview of events
			\item Identification of frequent patterns
		\end{itemize}
	\end{block}
	\begin{block}{Statistical analysis}
		\begin{itemize}
			\item E.g. event rate:
				\begin{itemize}
					\item Average: 1-2 events per minute
					\item Peaks: Up to 900 events per minute
				\end{itemize}
		\end{itemize}
	\end{block}
	\begin{block}{Pattern identification}
		\begin{itemize}
			\item E.g. temporal correlation (next slide)
		\end{itemize}
	\end{block}
	% \begin{block}{Statistics and findings}
		% \begin{itemize}
			% \item Average: 1-2 events per minute
			% \item Peaks: 900 events per minute ($\to$ bursts)
			% \item Top 5 event types responsible for more than half of all events \\($\to$ compression should be helpful)
			% \item Frequent pattern e.g.
				% \begin{itemize}
					% \item Service down/up due to (repeated) short ISP problems
				% \end{itemize}
		% \end{itemize}
	% \end{block}
\end{frame}

\begin{frame}
	\frametitle{Event Pattern Analysis: Temporal Correlation}
	\vspace{-0.4cm}\begin{center}\includegraphics[scale=0.35]{matrix_top13.png}\end{center}
\end{frame}


\subsection{Analysis of Existing Approaches}
\begin{frame}
	\frametitle{Analysis of Existing Approaches}
	\begin{block}{Correlation Approach}
		\begin{itemize}
			\item Decisions of the correlation engine should be
				reproducible and understandable for humans
				\begin{itemize}
					\item[$\Rightarrow$] Rule-based approach most suitable
				\end{itemize}
			\item Finite-state machine would be suitable for some patterns
				\begin{itemize}
					\item[$\Rightarrow$] Allow regular expressions on events
				\end{itemize}
		\end{itemize}
	\end{block}

	\begin{block}{Existing Software}	
		\begin{itemize}
			\item No suitable software found
			\item Many concepts can be reused, e.g.:
				\begin{itemize}
					\item Dynamic contexts% for representation of internal state (from SEC and others)
					\item Combination of simple building blocks into powerful
						rules
					% \item XML rules (OSSIM, OSSEC, others)
				\end{itemize}
		\end{itemize}
	\end{block}
\end{frame}

\subsection{Specification and Implementation}

\begin{frame}
	\frametitle{Specification}
	\begin{block}{Requirements and design decisions}
		\begin{itemize}
			\item Rule language should be easy to learn
				\begin{itemize}
					\item[$\Rightarrow$] XML based rules
				\end{itemize}
			% \item Scalability: large number of rules should not result in confusion
				% \begin{itemize}
					% \item[$\Rightarrow$] Separation of correlation rules into independent groups
				% \end{itemize}
			\item Local and global correlation should be possible
				\begin{itemize}
					\item[$\Rightarrow$] Tree of homogeneous correlation nodes
				\end{itemize}
		\end{itemize}
	\end{block}
	\begin{block}{Implementation}
		\begin{itemize}
			\item Functional programming to build rule functions from simple
				components
			\item Rapid prototyping was valued higher than execution speed
				\begin{itemize}
					\item[$\Rightarrow$] Implemented in Python
				\end{itemize}
			% \item Web interfaces as front-ends
		\end{itemize}
	\end{block}
\end{frame}

\begin{frame}
	\frametitle{Implementation: Node Tree}
	\begin{center}\includegraphics[scale=0.35]{ce_tree.pdf}\end{center}
\end{frame}

\begin{frame}
	\frametitle{Implementation: Functional Overview}
	\begin{center}\includegraphics[scale=0.8]{ce_node.pdf}\end{center}
\end{frame}

\subsection{Testing and Evaluation}

\begin{frame}
	\frametitle{Testing and Evaluation}
	\begin{block}{Testing and evaluation methods}
		\begin{itemize}
			\item Profiling
				% \begin{itemize}
					% \item[$\Rightarrow$] Lead to several minor improvements
				% \end{itemize}
			\item Unit tests
				% \begin{itemize}
					% \item[$\Rightarrow$] Currently 66 tests
				% \end{itemize}
			\item Sanity checks
				% \begin{itemize}
					% \item[$\Rightarrow$] E.g. event balance
				% \end{itemize}
			\item Evaluation with random events
			\item Evaluation with replayed real events
		\end{itemize}
	\end{block}
\end{frame}

\section{Conclusions}
\subsection{Conclusions}
\begin{frame}
	\frametitle{Conclusions}
	\begin{block}{Strengths}
		\begin{itemize}
			\item Functional programming allows to build rule functions at
				startup
				\begin{itemize}
					\item[$\Rightarrow$] No need for (slow and hard to debug)
						\texttt{eval()} during operation
		 		\end{itemize}
			% \item Tree structure with homogeneous nodes
				% \begin{itemize}
					% \item[$\Rightarrow$] Simple operations (filtering,
						% enrichment) close to event source, more complex
						% correlation centrally
					% \item[$\Rightarrow$] No need for separate agent and master
				% \end{itemize}
			\item Regular expressions to match patterns suitable for FSMs
				\begin{itemize}
					\item[$\Rightarrow$] As powerful as FSMs, but better known
						and easier to use
				\end{itemize}
			\item Cache automatically decides how long to keep an event
		\end{itemize}
	\end{block}
	\begin{block}{Aptitude for real-world events}
		\begin{itemize}
			\item Events of one month can be processed in $< 10$ minutes
				\begin{itemize}
					\item[$\Rightarrow$] Real-time operation no problem
				\end{itemize}
			\item Simple compression can reduce the event volume by  $> 50\%$
			\item Successful detection of complex patterns
				\begin{itemize}
					\item[$\Rightarrow$] E.g. detection of successful failover
						transition with regexp
				\end{itemize}
		\end{itemize}
	\end{block}
\end{frame}

\subsection{Outlook}
\begin{frame}
	\frametitle{Outlook}
	\begin{block}{Future development}
		\begin{itemize}
			\item Support for rule creation (e.g. pattern mining, GUI tools)
			\item Central rule database with target scopes for each rule group
				\begin{itemize}
					\item[$\Rightarrow$] ``Rule applies to all hosts in country X'' 
				\end{itemize}
			\item Reducing complexity
		\end{itemize}
	\end{block}
\end{frame}

\section{Demo}

\subsection{Example Problem}

\begin{frame}
	\frametitle{Demo: Example Problem}
	\begin{block}{Example: Irrelevant IP theft events}
		\begin{itemize}
			\item Standby firewall becomes master, while primary firewall is
				still up

			\item Firewalls detect second host with same IP address
				\begin{itemize}
					\item[$\Rightarrow$] Events indicating IP address theft
				\end{itemize}

			\item Duplicate master is also detected
				\begin{itemize}
					\item Corresponding event often generated after IP theft
						event
				\end{itemize}

			\item It is sufficient to solve the root problem (duplicate master)
				\begin{itemize}
					\item[$\Rightarrow$] IP address theft events are of no interest
				\end{itemize}
		\end{itemize}
	\end{block}
\end{frame}

\subsection{Correlation Approach}

\begin{frame}
	\frametitle{Demo: Correlation Approach}
	\begin{block}{Correlation behaviour}
		\begin{itemize}
			\item Event indicating a duplicate master
				\begin{itemize}
					\item[$\Rightarrow$] Represent this knowledge as context
					\item[$\Rightarrow$] Suppress IP theft events from same
						host during last minute
				\end{itemize}
			\item As long as context exists, suppress further IP theft events
			\item Event indicating duplicate master is gone
				\begin{itemize}
					\item[$\Rightarrow$] Remove context
				\end{itemize}
		\end{itemize}
	\end{block}
	\begin{block}{Demo}
		\begin{itemize}
			\item Real events with anonymized host names
			\item One correlation node
				\begin{itemize}
					\item First run: without correlation rules 
					\item Second run: rules with behaviour explained above
				\end{itemize}
		\end{itemize}
	\end{block}
\end{frame}

\section*{Questions}
\subsection*{Questions}
\begin{frame}
	\frametitle{Questions?}
	\begin{center}	
		Thank you for your attention.
	\end{center}
\end{frame}

\end{document}

