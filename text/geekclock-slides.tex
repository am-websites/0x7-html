\documentclass{beamer}		%Artikel Dokumentvorlage
\usepackage[ngerman]{babel}	%neue deutsche Rechtschreibung
\usepackage[latin1]{inputenc}	%latin1 Zeichensatz: damit ae,oe,ue direkt in den Text geschrieben werden koennen
\usepackage{helvet}


\usetheme{Warsaw}


\title[Geekclock -- Code and Hardware Walkthrough]{Geekclock}
\subtitle{Code and Hardware Walkthrough}
\author{Andreas M\"uller}
\date{}
\institute{\\\large Chaos Singularity 2007}

\AtBeginSection[]{\frame{\tableofcontents[current,sectionstyle=show/shaded,subsectionstyle=show/show/hide]}}
\setbeamercovered{transparent=50}

\begin{document}

\frame[plain]{\titlepage}

\setcounter{tocdepth}{2}
\frame[plain]{\tableofcontents}

\section{Elektronik}
\subsection{Stromkreis}
\begin{frame}
	\frametitle{Stromkreis}
	\begin{minipage}{3.5cm}
		\begin{center}\includegraphics[scale=0.2]{../doc/pics/circuit.png}\end{center}
	\end{minipage}
	\begin{minipage}{6.5cm}
		\begin{itemize}
			\item Strom fliesst nur im geschlossenen Kreis
			\item hier sind LED und Widerstand in Serie
			\begin{itemize}
				\item gleicher Strom fliesst durch beide Elemente
				\item Spannungen �ber den Elementen sind unterschiedlich
				\item bei Parallelschaltung w�re es umgekehrt 
			\end{itemize}
		\end{itemize}
	\end{minipage}
\end{frame}

\subsection{Gesetz von Ohm}
\begin{frame}
	\frametitle{Ohm'sches Gesetz}
	Proportionalit�t zwischen Widerstand $R$, Strom $I$ durch $R$, und Spannung $U$ �ber $R$:
	\[U=R*I\]
	Beispiel von der letzten Folie -- ben�tigter Widerstand?
	\begin{itemize}
		\item Spannung �ber Widerstand (Batteriespannung - LED bias): $U=4.5V-2V=2.5V$
		\item Strom: LED soll ca 10mA haben
		\item Widerstand: \[R=\frac{U}{I}=\frac{2.5V}{10mA}=\frac{2.5V}{0.01A}=250 \Omega\]
	\end{itemize}
\end{frame}

\subsection{Knotenpunkt- und Maschenregel}
\begin{frame}
	\frametitle{Knotenpunkt- und Maschenregel}
	\begin{minipage}{6cm}
		Kirchhoff'sche Gesetze:
		\begin{itemize}
			\item Knotenregel: Summe aller Str�me in einem Knoten ist Null ($\rightarrow$ es gehen keine Elektronen verloren)
			\item Maschenregel: Summe aller Spannungen in einer Masche ist Null ($\rightarrow$ es f�llt �ber einer idealen Leitung keine Spannung ab)
		\end{itemize}
	\end{minipage}
	\begin{minipage}{4cm}
		\begin{center}\includegraphics[scale=0.25]{pics/circuit_example.png}\end{center}
	\end{minipage}
\end{frame}

\subsection{Komponenten}
\begin{frame}
	\frametitle{Widerstand}
	\begin{itemize}
		\item Symbol: $R$
		\item Schaltzeichen: \includegraphics[scale=0.15]{pics/r.jpg}
		\item Kenngr�sse: Widerstand mit Einheit Ohm ($\Omega$)
		\item Spannung �ber Widerstand ist proportional zu Strom
		\item Farbcodierung gibt Widerstandswert an 
		%\item Einbaurichtung egal
	\end{itemize}
	\begin{center}\includegraphics[scale=0.5]{pics/r_photo.jpg}\end{center}
\end{frame}

\begin{frame}
	\frametitle{Kondensator}
	\begin{itemize}
		\item Symbol: $C$
		\item Schaltzeichen: \includegraphics[scale=0.15]{pics/c.jpg}
		\item Kenngr�sse: Kapazit�t mit Einheit Farad ($F$)
		\item Schaltzeichen f�r Elektrolytkondensatoren: \includegraphics[scale=0.15]{pics/cpol.jpg}
		\item speichert Strom / stabilisiert Spannung
		%\item Einbaurichtung bei Elektrolytkondensatoren relevant
		%\item zeitliche �nderung der Spannung �ber dem Kondensator proportional zum Stromfluss:  \[i=C\frac{\mathrm{d} u}{\mathrm{d} t}\]
		\item Werte sind meist direkt aufgedruckt
	\end{itemize}
	\begin{center}\includegraphics[scale=0.2]{pics/c_photo.jpg}\end{center}
\end{frame}

\begin{frame}
	\frametitle{Diode und LED}
	\begin{itemize}
		\item Symbol: $D$
		\item Schaltzeichen: \includegraphics[scale=0.15]{pics/d.jpg} bzw. \includegraphics[scale=0.15]{pics/led.jpg} (LED)
		\item Dioden lassen Strom nur in eine Richtung durch $\to$ Einbaurichtung (Polarit�t) beachten
		\item in der Geekclock als Anzeige (LED) und Verpolungsschutz
		\item LED: Light Emitting Diode
	\end{itemize}
	\begin{center}\includegraphics[scale=0.4]{pics/led_photo.jpg}\end{center}
\end{frame}

\begin{frame}
	\frametitle{Quarz}
	\begin{itemize}
		\item Symbol: $Q$
		\item Schaltzeichen: \includegraphics[scale=0.15]{pics/q.jpg}
		\item liefert sehr stabile Referenzfrequenz
		\item Funktionsweise: Quarzpl�ttchen mit angelegten Elektroden:
		\begin{itemize}
			\item Quarz verbiegt sich beim Anlegen einer Spannung
			\item Spannung weg $\to$ Deformation umgekehrt $\to$ Spannung wird produziert
			\item positive R�ckkoppelung nur bei Resonanzfrequenz und Harmonischen
		\end{itemize}
	\end{itemize}
	\begin{center}\includegraphics[scale=0.8]{pics/q_photo.jpg}\end{center}
\end{frame}

\section{Using Microcontrollers}
\subsection{What is an MCU?}
\begin{frame}
	\frametitle{What is a microcontroller?}
	\begin{itemize}
		\item Wikipedia: A microcontroller (or MCU) is a computer-on-a-chip. It is a type of microprocessor emphasizing self-sufficiency and cost-effectiveness, in contrast to a general-purpose microprocessor (the kind used in a PC).
		\item RAM, ROM, memory and a CPU are, along with various peripherals, all contained on a single chip, which can be programmed to fulfill a specific task.
	\end{itemize}
\end{frame}
\subsection{ATMega8 features}
\begin{frame}
	\frametitle{ATMega8}
	\begin{itemize}
		\item RISC Microcontroller, max 16MHz
		\item 23 I/O lines
		\item lots of integrated peripherals
		\begin{itemize}
			\item timers
			\item AD converters
			\item PWM
			\item internal or external oscillator possible
		\end{itemize}
		\item sleep mode support
		\item In-System Programmable Flash memory
	\end{itemize}
\end{frame}
\subsection{Differences from coding on a PC}
\begin{frame}
	\frametitle{MCU coding peculiarities}
	\begin{itemize}
		\item less powerful hardware
		\item in our case
		\begin{itemize}
			\item 32kHz core frequency (up to 16MHz would be possible)
			\item 1KB SRAM
			\item 8KB Flash memory
			\item[] ... ought to be enough for everyone
		\end{itemize}
		\item no FPU
	\end{itemize}
\end{frame}
\begin{frame}
	\frametitle{MCU coding peculiarities (continued)}
	\begin{itemize}
		\item no OS
		\begin{itemize}
			\item only one process
			\item no virtual memory, etc
			\item hard real time is possible
			\item avr-libc provides some functions
		\end{itemize}
		\item no printf
		\item no easy way to tell if an error is in software or in hardware
		\item programs are usually designed to never reach an end 
	\end{itemize}
\end{frame}
\begin{frame}
	\frametitle{some advice for efficient coding}
	\begin{itemize}
		\item use \texttt{gcc} with $\texttt{-Os}$ (\texttt{-O2} and optimize for size)
		\item don't use 32bit integers, when you only need 8bit ($\to$ use \texttt{uint8\_t} or \texttt{int8\_t}) [demo]
		\item condition checks are preferable to expressions with modulo operations or multiplications
		\item avoid floating point variables and functions (\texttt{sin(), sqrt(), ..})
		\item keep variable count low (even if the SRAM is big enough -- if you have only a few variables, they can always stay in the registers)
		\item there is usually no need to code in assembler
		\item don't worry ... 32kHz is more than it might seem
	\end{itemize}
\end{frame}


\section{Geekclock Hardware}
\subsection{Hardware Overview}
\begin{frame}
	\frametitle{Geekclock Hardware Overview}
	\begin{itemize}
		\item core: ATMega8 MCU 
		\item clock from 32kHz crystal (low frequency to save power)
		\item 6 LEDs to show time in binary
		\item button to control clock
		\item diode to protect MCU from wrong polarity
		\item interface for programming via LPT cable or USB programmer
	\end{itemize}

\end{frame}
\subsection{Circuit diagram}
\begin{frame}[plain]
	\begin{center}\includegraphics[scale=0.8]{../hardware/simplified/schema.png}\end{center}
\end{frame}

\section{Geekclock Software}
\subsection{Software concept}
\begin{frame}
	\frametitle{Software concept}
	\begin{itemize}
		\item hardware timer generates interrupt each second
		\item time is updated in interrupt routine
		\item button generates interrupt
		\item time is shown in main routine after button was pressed
	\end{itemize}
\end{frame}
\subsection{Structure Overview}
\begin{frame}
	\frametitle{Structure overview}
	\begin{itemize}
		\item \texttt{geekclock.c}: interrupts, main-routine
		\item \texttt{lowlevel.c}: initialisation (Timer, Ports), lowlevel functions %(\texttt{delay\_ms()}, \texttt{button\_down()}, \texttt{set\_leds()})
		\item \texttt{datetime.c}: calendar functions, time functions 
		\item \texttt{led.c}: LED control, effects
	\end{itemize}
\end{frame}
\subsection{Code Walkthrough}
\begin{frame}
	\frametitle{Code}
	\begin{center}(Code)\end{center}
\end{frame}

\appendix
\section*{Questions}
\begin{frame}
	\frametitle{Questions}
		\begin{center}Questions?\end{center}
\end{frame}


\end{document}
